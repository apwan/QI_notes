\documentclass{article}
\usepackage{amsmath,amsfonts,amsthm,amssymb}
\usepackage{setspace}
\usepackage{fancyhdr}
\usepackage{lastpage}
\usepackage{extramarks}
\usepackage{extarrows}
\usepackage{chngpage}
\usepackage{soul,color}
\usepackage{mathrsfs}
\usepackage[linesnumbered,boxed]{algorithm2e}  %for using box
\usepackage{graphicx,float,wrapfig}     %for inserting figures etc.
\usepackage{verbatim}   %for using the environment of comment
\usepackage{multicol}    %for using the environment of multicolumn
\usepackage{tikz}           %for convenient drawing
\usetikzlibrary{arrows,decorations.pathmorphing,backgrounds,positioning,fit,petri}
\usepackage{pdfpages}
\usepackage[pdftex,colorlinks=true,linkcolor=black,urlcolor=blue]{hyperref}

%BASIC INFO
\newcommand{\StudentName}{Wu Yijie}
\newcommand{\StudentClass}{JK30}
\newcommand{\StudentNumber}{2013011314}
\newcommand{\Class}{Quantum Information}
\newcommand{\ClassInstructor}{Xiongfeng Ma}


%%%% In case you need to adjust margins:
\topmargin=-0.75in      %
\evensidemargin=0in     %
\oddsidemargin=0in      %
\textwidth=6.5in        %
\textheight=9.5in       %
\headsep=0.10in         %

%%%% Setup the header and footer
\pagestyle{fancy}                                                       %
                     %
\chead{\Title\quad\firstxmark}  %
\rhead{Page\ \thepage\ of\ \protect\pageref{LastPage}}                                                     %
\lfoot{\lastxmark}                                                      %
\cfoot{}                                                                %
\rfoot{}             %
\renewcommand\headrulewidth{0.4pt}                                      %
\renewcommand\footrulewidth{0pt}                                      %

%%%%%%%%%%%%%%%%%%%%%%%%%%%%%%%%%%%%%%%%%%%%%%%%%%%%%%%%%%%%%
% Some tools
\newcommand{\enterProblemHeader}[1]{\nobreak\extramarks{#1}{#1 continued on next page\ldots}\nobreak%
                                    \nobreak\extramarks{#1 (continued)}{#1 continued on next page\ldots}\nobreak}%
\newcommand{\exitProblemHeader}[1]{\nobreak\extramarks{#1 (continued)}{#1 continued on next page\ldots}\nobreak%
                                   \nobreak\extramarks{#1}{}\nobreak}%

\newcommand{\homeworkProblemName}{}%
\newcounter{homeworkProblemCounter}%
\newenvironment{hwPro}[1][Problem \arabic{homeworkProblemCounter}]%
  {\stepcounter{homeworkProblemCounter}%
   \renewcommand{\homeworkProblemName}{#1}%
   \section*{\homeworkProblemName}%
   \enterProblemHeader{\homeworkProblemName}}%
  {\exitProblemHeader{\homeworkProblemName}}%

\newcommand{\homeworkSectionName}{}%
\newlength{\homeworkSectionLabelLength}{}%
\newenvironment{hwSec}[1]%parts of homework problem
  {% We put this space here to make sure we're not connected to the above.

   \renewcommand{\homeworkSectionName}{#1}%
   \settowidth{\homeworkSectionLabelLength}{\homeworkSectionName}%
   \addtolength{\homeworkSectionLabelLength}{0.2in}%
   \changetext{}{-\homeworkSectionLabelLength}{}{}{}%
   \subsection*{\homeworkSectionName}%
   \enterProblemHeader{\homeworkProblemName\ [\homeworkSectionName]}}%
  {\enterProblemHeader{\homeworkProblemName}%

   % We put the blank space above in order to make sure this margin
   % change doesn't happen too soon.
   \changetext{}{+\homeworkSectionLabelLength}{}{}{}}%

\newcommand{\Answer}{\ \\\textbf{Answer:} }   %note first letter 'A' capital!
\newcommand{\Acknowledgement}[1]{\ \\{\bf Acknowledgement:} #1}
\newcommand{\wtM}{\textcolor{white}{M}}  %for indent in before paragraphs
\newcommand{\ud}{\mathrm{d}} %for derivative
\newcommand{\cm}{\mathrm{\,cm}} %centimeter
%%%%%%%%%%%%%%%%%%%%%%%%%%%%%%%%%%%%%%%%%%%%%%%%%%%%%%%%%%%%%

\newcommand{\Title}{Week 2}
\newcommand{\DueDate}{September, 2015}


%% Useful short-cut
\newcommand{\Tr}{\mathrm{Tr}}
\newcommand{\bfone}{\mathbf{1}}
\newcommand{\bra}[1]{\langle #1\vert}
\newcommand{\ket}[1]{\vert #1\rangle}
\newcommand{\braket}[2]{\langle #1 \vert #2\rangle}
\newcommand{\ketbra}[2]{\vert #1\rangle \langle #2\vert}


%%%%%%%%%%%%%%%%%%%%%%%%%%%%%%%%%%%%%%%%%%%%%%%%%%%%%%%%%%%%%

\lhead{\StudentName\quad \StudentNumber}
% Make title
\title{\textmd{\bf \Class\\ \Title}\\{\large Instructed by \textit{\ClassInstructor}}\\\normalsize\vspace{0.1in}\small{Due\ on\ \DueDate}}
\date{}
\begin{document}

\begin{spacing}{1.2}
\author{\textbf{\StudentName}\qquad\StudentClass\quad\StudentNumber}
\maketitle \thispagestyle{empty}
\section{Last Lecture}
Bell's inequality. Loopholes.

Non-local correlation (entanglement) can not be used to communicate. Signaling is stronger.

Hardmard transformation $H: \ket{0}\to \ket{+}, \ket{1}\to\ket{-}$.

C-NOT: diag($I_2,\sigma_x$).

Four Bell states: $\ket{00}\pm\ket{11},\ket{01}\pm\ket{10}$.

\section{Quantum Circuit}
$H$ and C-NOT: $\ket0 \otimes \ket1 \to \ket{00}+\ket{11}$.

\subsection{Bell-state measure (BSM)}
$\ket{00}\pm\ket{11}\to_{CNOT}\ket{\pm}\ket0$. $\ket{01}\pm\ket{10}\to_{CNOT}\ket{\pm}\ket{1}$.

\subsection{Teleportation (quantum channel)}
Super dense coding: e.g., transfer 2 (classical) bit information by using the quantum channel only once (between Earth and Mars)

Alice can perform $\sigma_x,\sigma_z, \sigma_z\sigma_x=\sigma_y$ on her share of qubit (start from $\ket{00}+\ket{11}$).

one qubit + one ebit (entangled qubit) $\geq$ one cbit (classical bit), help increase channel capacity.

two cbit + 1ebit $\geq$ one qubit.

Without quantum channel, use classical channel. For pure $\ket{\psi} = a\ket0 +b\ket1$, just transfer $a$ (require too many bit).

We want to transfer faithfully (assume there are already some ebits, e.g. $\ket\phi =\ket{00}+\ket{11}$). Perform BSM on \begin{gather*}
\ket{psi}\ket{\phi} = (\ket{00}+\ket{11})(a\ket0 + b\ket1) - b\ket{001} - a\ket{110} + (\ket{01}+\ket{10})(a\ket1 + b\ket{0}) - b\ket{010}-a\ket{101}\\
=\sum(\ket{00}\pm\ket{11})(a\ket0 \pm b\ket1) + \sum(\ket{01}\pm \ket{10})(a\ket1 \pm b\ket0).
\end{gather*}
Strong symmetry. The BSM will project the joint state to one of the Bell-state with Bob's share of qubit containing the information of $a,b$. Thus the qubit can be transfered faithfully when Alice project to $\ket{00}+\ket{11}$. Remote state preparation if Alice project to other state.

Simper way if Alice knows $a,b$. Alice: $\sigma_z$, Bob: $\sigma_x/ \sigma_I$. Alice: X, Bob: $\sigma_z$.

Exercise: Perform $H\otimes H$ to all the four Bell states (two unchanged, two swapped).

$\ket{00}+\ket{11} = \ket{\psi\psi}+\ket{\psi^\perp \psi^\perp}$. Then Bob perform $\ketbra{\psi^\perp}{\psi}+\ketbra{\psi}{\psi^\perp}$ (depend on $a,b$).


Construct universal NOT gate. (Does not exist unless $a,b\in\mathbb{R}$!)  So there is failure rate.

If not real, $\ket{00}+\ket{11}\neq \ket{+i+i}+\ket{-i-i}$, but still one of the Bell-state.

$\ket{01}-\ket{10}$ can not contain $\ket{\psi\psi}$, but contain $\ket{\psi \psi^\perp}$. Spin coupling: Singlet, triplet



\end{spacing}


\end{document}

