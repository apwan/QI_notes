\documentclass{article}
\usepackage{amsmath,amsfonts,amsthm,amssymb}
\usepackage{setspace}
\usepackage{fancyhdr}
\usepackage{lastpage}
\usepackage{extramarks}
\usepackage{extarrows}
\usepackage{chngpage}
\usepackage{soul,color}
\usepackage{mathrsfs}
\usepackage[linesnumbered,boxed]{algorithm2e}  %for using box
\usepackage{graphicx,float,wrapfig}     %for inserting figures etc.
\usepackage{verbatim}   %for using the environment of comment
\usepackage{multicol}    %for using the environment of multicolumn
\usepackage{tikz}           %for convenient drawing
\usetikzlibrary{arrows,decorations.pathmorphing,backgrounds,positioning,fit,petri}
\usepackage{pdfpages}
\usepackage[pdftex,colorlinks=true,linkcolor=black,urlcolor=blue]{hyperref}

%BASIC INFO
\newcommand{\StudentName}{Wu Yijie}
\newcommand{\StudentClass}{JK30}
\newcommand{\StudentNumber}{2013011314}
\newcommand{\Class}{Quantum Information}
\newcommand{\ClassInstructor}{Xiongfeng Ma}


%%%% In case you need to adjust margins:
\topmargin=-0.75in      %
\evensidemargin=0in     %
\oddsidemargin=0in      %
\textwidth=6.5in        %
\textheight=9.5in       %
\headsep=0.10in         %

%%%% Setup the header and footer
\pagestyle{fancy}                                                       %
                     %
\chead{\Title\quad\firstxmark}  %
\rhead{Page\ \thepage\ of\ \protect\pageref{LastPage}}                                                     %
\lfoot{\lastxmark}                                                      %
\cfoot{}                                                                %
\rfoot{}             %
\renewcommand\headrulewidth{0.4pt}                                      %
\renewcommand\footrulewidth{0pt}                                      %

%%%%%%%%%%%%%%%%%%%%%%%%%%%%%%%%%%%%%%%%%%%%%%%%%%%%%%%%%%%%%
% Some tools
\newcommand{\enterProblemHeader}[1]{\nobreak\extramarks{#1}{#1 continued on next page\ldots}\nobreak%
                                    \nobreak\extramarks{#1 (continued)}{#1 continued on next page\ldots}\nobreak}%
\newcommand{\exitProblemHeader}[1]{\nobreak\extramarks{#1 (continued)}{#1 continued on next page\ldots}\nobreak%
                                   \nobreak\extramarks{#1}{}\nobreak}%

\newcommand{\homeworkProblemName}{}%
\newcounter{homeworkProblemCounter}%
\newenvironment{hwPro}[1][Problem \arabic{homeworkProblemCounter}]%
  {\stepcounter{homeworkProblemCounter}%
   \renewcommand{\homeworkProblemName}{#1}%
   \section*{\homeworkProblemName}%
   \enterProblemHeader{\homeworkProblemName}}%
  {\exitProblemHeader{\homeworkProblemName}}%

\newcommand{\homeworkSectionName}{}%
\newlength{\homeworkSectionLabelLength}{}%
\newenvironment{hwSec}[1]%parts of homework problem
  {% We put this space here to make sure we're not connected to the above.

   \renewcommand{\homeworkSectionName}{#1}%
   \settowidth{\homeworkSectionLabelLength}{\homeworkSectionName}%
   \addtolength{\homeworkSectionLabelLength}{0.2in}%
   \changetext{}{-\homeworkSectionLabelLength}{}{}{}%
   \subsection*{\homeworkSectionName}%
   \enterProblemHeader{\homeworkProblemName\ [\homeworkSectionName]}}%
  {\enterProblemHeader{\homeworkProblemName}%

   % We put the blank space above in order to make sure this margin
   % change doesn't happen too soon.
   \changetext{}{+\homeworkSectionLabelLength}{}{}{}}%

\newcommand{\Answer}{\ \\\textbf{Answer:} }   %note first letter 'A' capital!
\newcommand{\Acknowledgement}[1]{\ \\{\bf Acknowledgement:} #1}
\newcommand{\wtM}{\textcolor{white}{M}}  %for indent in before paragraphs
\newcommand{\ud}{\mathrm{d}} %for derivative
\newcommand{\cm}{\mathrm{\,cm}} %centimeter
%%%%%%%%%%%%%%%%%%%%%%%%%%%%%%%%%%%%%%%%%%%%%%%%%%%%%%%%%%%%%

\newcommand{\Title}{Week 2}
\newcommand{\DueDate}{September, 2015}


%% Useful short-cut
\newcommand{\Tr}{\mathrm{Tr}}
\newcommand{\bfone}{\mathbf{1}}
\newcommand{\bra}[1]{\langle #1\vert}
\newcommand{\ket}[1]{\vert #1\vert}
\newcommand{\braket}[2]{\langle #1 \vert #2\rangle}
\newcommand{\ketbra}[2]{\vert #1\rangle \langle #2\vert}


%%%%%%%%%%%%%%%%%%%%%%%%%%%%%%%%%%%%%%%%%%%%%%%%%%%%%%%%%%%%%

\lhead{\StudentName\quad \StudentNumber}
% Make title
\title{\textmd{\bf \Class\\ \Title}\\{\large Instructed by \textit{\ClassInstructor}}\\\normalsize\vspace{0.1in}\small{Due\ on\ \DueDate}}
\date{}
\begin{document}

\begin{spacing}{1.2}
\author{\textbf{\StudentName}\qquad\StudentClass\quad\StudentNumber}
\maketitle \thispagestyle{empty}
\section{Homework 1 due on Oct. 12}
Hand in to MMW-S323 mail box 1st floor.

Problem 3(b) add ``with 100\% probability".

Problem 2(b) updated: function of matrix $f(\sigma) = \sum_i f(\lambda_i) \ketbra{\lambda_i}{\lambda_i} $.

Contents: Purification, joint measure, Bell basis/states, C-NOT, BSM, Tomography, Bell's inequality.

\section{Tomography}
$\rho = \sum_x p(x)\ketbra{x} \otimes f(x)$, e.g. $\frac12 \left[ \ketbra{0} \otimes \rho_0 + \ketbra{1}\otimes \rho_1\right]$.

Measure $x,y,z$ for $\rho = \frac12 (I+\vec{P}\cdot\vec{\sigma})$. $P_i = \Tr(\rho\sigma_i)(i = x,y,z)$.
\begin{gather*}
\rho = \frac12 \left( \Tr(\rho I)I + \sum_{i=x,y,z}\Tr(\sigma_i \rho)\sigma_i\right)
\end{gather*}

\subparagraph{two-qubit system.}
Use $\sigma_i\otimes\sigma$ to measure $\rho_{AB}$. 

For $\frac{1}{\sqrt2}(\ket{00}_{AB}\pm \ket{11}_{AB}), \frac{1}{\sqrt2}(\ket{01}\pm\ket{10})$ (can not be written as tensor product),

$\rho_A = \frac{1}{2}[1 0; 0 1] = \rho_B, \rho_{AB} = \rho_A\otimes\rho_B = \left(\begin{array}{cccc}
&&&\\
&&&\\
&&&\\
&&&
\end{array}\right)_{4\times 4} = \sum_{i,j\in\{I,x,y,z\}}P_{ij}\sigma_i\otimes\sigma_j$.


Beccause $\Tr{\rho_{AB}} = \sum_{i,j}P_{ij}\Tr{\sigma_i\otimes\sigma_j} = P_{II}\Tr(\sigma_I)\Tr(\sigma_I) = 4P_{II}=1$, $P_{II} = \frac14$.

$P_{ij} = \Tr(\sigma_{i'j'}P_{i'j'}(\sigma_{i'}\otimes\sigma_{j'})(\sigma_i\otimes\sigma_j)) = \Tr(\rho\sigma_i\otimes\sigma_j)$.

\subparagraph{n-qubit system.}
$\bigotimes_{v_i=\in\{I,x,y,z\}} \sigma_{v_i}, \rho = \sigma_{v_i}\Tr(\rho\bigotimes_{v_i}\sigma_{v_i})\bigotimes(\sigma_{v_i}).$
$3^n$ measurement should be performed, but the DOF is less, so this tomography is not optimal. 

\subparagraph{Individual measurement:} $4^n$ terms (nothing for $I$).
\subparagraph{Joint measurement (BSM: Bell-state measurement):} $\phi^{\pm} = \ket{00}\pm\ket{11}, \psi^{\pm} = \ket{01}\pm\ket{10}$

Hamard $H = \frac{1}{\sqrt2}[1, 1; 1, -1]$ operation, C-NOT operation (control-NOT, flip target qubit when input control qubit $\ket{1}$, e.g. $\ket{1}\ket{0}\mapsto\ket{1}\ket{1})$) is $4\times 4$ unitary matrix:
\begin{gather*}
\left(\begin{array}{cccc}
1&0&0&0\\
0&1&0&0\\
0&0&0&1\\
0&0&1&0
\end{array}\right).
\end{gather*}
$\ket{00}+\ket{11}\underrightarrow{C-NOT}\ket{+}\ket{0} \underrightarrow{H\times I} \ket{0}\ket{0}$, $\ket{00}-\ket{11}\underrightarrow{C-NOT, H\otimes I} \ket{1}\ket{0}, \ket{01}+\ket{10}\mapsto \ket{0}\ket{1}, \ket{01}-\ket{10}\mapsto \ket{1}\ket{1}, \ket{00}\mapsto\ket{+}\ket{0}$. Then measure in $z$-basis.

\subparagraph{Purification.} Given $\rho_A = \frac{1}{2}(\ketbra{0}+\ketbra{1})$, find $\Psi_{AB}$ s.t. $\Tr_{AB} = \rho_A$. Answer: $\ket{00}+\ket{11}$, not unique, but all purification can be linked by linear operation. e.g.
$ \left(\begin{array}{cc}
\frac49 &\frac{1}{100}\\
\frac{1}{100}&\frac59\\
\end{array}\right)$

For diagonal terms, $\sqrt{\frac49}\ket{0}\ket{0}+\sqrt{\frac59}\ket{1}$. For general matrix, extends $\rho_A$ to pure $\rho_{AB}$.
Choose a basis such that $\ket{\psi_{AB}} = \sum \sqrt{x_i}\ket{i}_A\ket{i}_B$ is diagonalized.

Denote eigenvalues of $\rho$ as $\lambda_0,\lambda_1$, eigen states $\ket{\psi_0}, \ket{\psi_1}$. $\rho' = \mathrm{diag}(\lambda_0,\lambda_1)$. $\rho = \lambda_0\ketbra{\psi_0}{\psi_0}+\lambda_1\ketbra{\psi_0}{\psi_0}$, then $\phi_{AB} = \lambda_0\ket{\psi_0}_A\ket{0}_B+\\lambda_1\ket{\psi_1}_A\ket{1}_B$. Comment: Given any mixed states, you can always find a larger Hilbert space such that the density matrix is purified (not unique, but there are relations between such purification).


\subparagraph{Schmidt decomposition.}
For pure $\ket{\psi}_{AB} = \sum\sqrt{p_i}\ket{i}\bra{i'}$, you can always find $\rho_A = \rho_B$ in all basis ($\rho_A = \Tr_B(\psi_{AB})$)

\subparagraph{Bell's inequality (John Bell 1964).} Refer to Jhon Preskill notes, Hoi-Kwong Lo notes. Local hidden variables can not express any quantum outcomes. Randomness is in the nature of quantum mechanics.


\section{XOR Games}

Separate boxes (e.g., 16 l.y. away), local referees each give Alice or Bob random variables. $x\in\{0,1\}\to [A] \to a\in\{0,1\}$; $y\in\{0,1\}\to [B]\to b\in\{0,1\}$. In order to win, Alice and Bob need to make sure $a\oplus b = x\cdot y$. No communication in the games, but before the game they can discuss the strategy.

Analysis: $\Pr[x\cdot y = 0] = \frac{3}{4}, \Pr[x\cdot y = 1] = \frac{1}{4}$.

 Random strategy can win with probability $\frac{1}{2}$. Always outputs same result, $a\oplus b = 0$, win with probability $\frac{3}{4}$.
 
Proof: No strategy achieves larger probability than $\frac34$. Maximize $S = \sum_{a,b,x,y} (-1)^{a\oplus b+ x\cdot y}\Pr[a,b\vert x,y]$. Note that $a$ can not depend on $y$ and $b$ can not depend on $x$, so $\Pr[a,b\vert x,y] = \Pr[a\vert x]\Pr[b\vert y]$.
\begin{eqnarray*}
S&=& \sum_{a,b,x,y}(-1)^{a+b+x\cdot y}\Pr[a\vert x]\Pr[b\vert y]\\
&=& \sum_{x,y} \sum_{a,x}(-1)^a\Pr[a\vert x]\sum_{b,y}(-1)^b \Pr[b\vert y](-1)^{xy}\\
&=& \sum_{a,x}(-1)^a \Pr[a\vert x]\sum_{b}(-1)^b\left[ \Pr[b\vert 0]+\Pr[b\vert 1](-1)^x \right]
\end{eqnarray*}

Case by case, denote the conditional expectations $A(x) = \Pr[a =0\vert x] -\Pr[a=1\vert x], B(y) = \Pr[b=0\vert y]-\Pr[b=1\vert y], A,B\in[-1,+1]$. Then $\Tr(\rho \sigma_z)\sim A$
\begin{enumerate}
\item $x\cdot y=0$: $\sum_{a,b} (-1)^{a+b}\Pr[a\vert x]\Pr[b\vert y] = \sum_{a}(-1)^a \Pr[a\vert x] \sum_b (-1)^b \Pr[b\vert y]$.
\item $x = y = 1$: $-\sum_{a}(-1)^a \Pr[a\vert 1] \sum_b (-1)^b \Pr[b\vert 1]$.
\end{enumerate} 
$S = A(0)B(0)+A(1)B(0)+A(0)B(1)-A(1)B(1) = A(0)[B(0)+B(1)]+A(1)[B(0)-B(1)] \leq 2$. Use probabilistic explanation, treat $A,B$ as random variables in $\{-1,1\}$. At least two terms are 0's. Or solve by linear optimization for continuous $A,B$ in $[-1,1]$.


Relate classical prob to quantum measurement. $x=0: A_0, \rho_A$. $\Tr(A_0 \rho_A) = A(0)$. Similar for $x=1, A_1; y=0, B_0; y=1, B_1$.

$\psi_{AB} = \ket{00}+\ket{11}$
$A_0 = \sigma_z, A_1 = \sigma_x, B_0 = \frac1{\sqrt2} (\sigma_x+\sigma_z), B_1 = \frac1{\sqrt2} (\sigma_x-\sigma_z)$.

The expectation $S = \langle A_0B_0 +A_0B_1+A_1B_0-A_1B_1\rangle$.
$\langle A_0 B_0\rangle = \Tr(\rho_{AB}\sigma_x\otimes \frac{\sigma_x+\sigma_z}{\sqrt{2}}) = \frac{1}{\sqrt2}\langle \sigma_x\otimes\sigma_x\rangle+\frac{1}{\sqrt2}\langle \sigma_x\otimes\sigma_z \rangle = \frac{1}{\sqrt2}+0$.

So $S = \frac{\sqrt2}{2}\times 3-(-\frac{\sqrt2}{2}) = 2\sqrt{2}>2$. (Why contradicts the classical results?)
\end{spacing}


\end{document}

