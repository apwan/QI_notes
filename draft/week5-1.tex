\documentclass{article}
\usepackage{amsmath,amsfonts,amsthm,amssymb}
\usepackage{setspace}
\usepackage{fancyhdr}
\usepackage{lastpage}
\usepackage{extramarks}
\usepackage{extarrows}
\usepackage{chngpage}
\usepackage{soul,color}
\usepackage{mathrsfs}
\usepackage[linesnumbered,boxed]{algorithm2e}  %for using box
\usepackage{graphicx,float,wrapfig}     %for inserting figures etc.
\usepackage{verbatim}   %for using the environment of comment
\usepackage{multicol}    %for using the environment of multicolumn
\usepackage{tikz}           %for convenient drawing
\usetikzlibrary{arrows,decorations.pathmorphing,backgrounds,positioning,fit,petri}
\usepackage{CJK}
\usepackage{pdfpages}
\usepackage[pdftex,colorlinks=true,linkcolor=black,urlcolor=blue]{hyperref}

%BASIC INFO
\newcommand{\StudentName}{Li Chenxing}
\newcommand{\StudentClass}{JK30}
\newcommand{\StudentNumber}{2013012479}
\newcommand{\Class}{Quantum Information}
\newcommand{\ClassInstructor}{Mile Gu}


%%%% In case you need to adjust margins:
\topmargin=-0.75in      %
\evensidemargin=0in     %
\oddsidemargin=0in      %
\textwidth=6.5in        %
\textheight=9.5in       %
\headsep=0.10in         %

%%%% Setup the header and footer
\pagestyle{fancy}                                                       %
                     %
\chead{\Title\quad\firstxmark}  %
\rhead{Page\ \thepage\ of\ \protect\pageref{LastPage}}                                                     %
\lfoot{\lastxmark}                                                      %
\cfoot{}                                                                %
\rfoot{}             %
\renewcommand\headrulewidth{0.4pt}                                      %
\renewcommand\footrulewidth{0pt}                                      %

%%%%%%%%%%%%%%%%%%%%%%%%%%%%%%%%%%%%%%%%%%%%%%%%%%%%%%%%%%%%%
% Some tools
\newcommand{\enterProblemHeader}[1]{\nobreak\extramarks{#1}{#1 continued on next page\ldots}\nobreak%
                                    \nobreak\extramarks{#1 (continued)}{#1 continued on next page\ldots}\nobreak}%
\newcommand{\exitProblemHeader}[1]{\nobreak\extramarks{#1 (continued)}{#1 continued on next page\ldots}\nobreak%
                                   \nobreak\extramarks{#1}{}\nobreak}%

\newcommand{\homeworkProblemName}{}%
\newcounter{homeworkProblemCounter}%
\newenvironment{hwPro}[1][Problem \arabic{homeworkProblemCounter}]%
  {\stepcounter{homeworkProblemCounter}%
   \renewcommand{\homeworkProblemName}{#1}%
   \section*{\homeworkProblemName}%
   \enterProblemHeader{\homeworkProblemName}}%
  {\exitProblemHeader{\homeworkProblemName}}%

\newcommand{\homeworkSectionName}{}%
\newlength{\homeworkSectionLabelLength}{}%
\newenvironment{hwSec}[1]%parts of homework problem
  {% We put this space here to make sure we're not connected to the above.

   \renewcommand{\homeworkSectionName}{#1}%
   \settowidth{\homeworkSectionLabelLength}{\homeworkSectionName}%
   \addtolength{\homeworkSectionLabelLength}{0.2in}%
   \changetext{}{-\homeworkSectionLabelLength}{}{}{}%
   \subsection*{\homeworkSectionName}%
   \enterProblemHeader{\homeworkProblemName\ [\homeworkSectionName]}}%
  {\enterProblemHeader{\homeworkProblemName}%

   % We put the blank space above in order to make sure this margin
   % change doesn't happen too soon.
   \changetext{}{+\homeworkSectionLabelLength}{}{}{}}%

\newcommand{\Answer}{\ \\\textbf{Answer:} }   %note first letter 'A' capital!
\newcommand{\Acknowledgement}[1]{\ \\{\bf Acknowledgement:} #1}
\newcommand{\wtM}{\textcolor{white}{M}}  %for indent in before paragraphs
\newcommand{\ud}{\mathrm{d}} %for derivative
\newcommand{\cm}{\mathrm{\,cm}} %centimeter
%%%%%%%%%%%%%%%%%%%%%%%%%%%%%%%%%%%%%%%%%%%%%%%%%%%%%%%%%%%%%

\newcommand{\Title}{Week 5}
\newcommand{\DueDate}{September, 2015}

\newcommand{\Tr}{\mathrm{Tr}}

%%%%%%%%%%%%%%%%%%%%%%%%%%%%%%%%%%%%%%%%%%%%%%%%%%%%%%%%%%%%%

\lhead{\StudentName\quad \StudentNumber}
% Make title
\title{\textmd{\bf \Class\\ \Title}\\{\large Instructed by \textit{\ClassInstructor}}\\\normalsize\vspace{0.1in}}
\date{}
\begin{document}
\begin{CJK}{UTF8}{gkai} %for chinese characters
\begin{spacing}{1.2}
\author{\textbf{\StudentName \quad李辰星}\qquad\StudentClass\quad\StudentNumber}
\maketitle \thispagestyle{empty}

\section{Shannon Entropy}

In this section, we consider the entropy on $z$ basis. Consider the density matrix $\rho$ of the classical coin.

\[\rho=c_0\vert 0 \rangle \langle 0 \vert+ c_1\vert 1 \rangle\langle 1 \vert.\]

Then the entropy of $\rho$ is

\[S(\rho)=-c_0\log c_0 -c_1 \log c_1.\]

Generally, we can define the entropy of the density matrix 

\[S(\rho)=-\sum_{\lambda}\lambda \log(\lambda)=-\Tr[\rho \log \rho].\]

\section{Mutual entropy}
{\color{white} 1}

We use $H(X)$ to denote the entropy of random variable $X$, and $H(X|Y)$ to denote the condition entropy of $X$ given $Y$. The definition of condition entropy will be proposed later on.

Suppose random variable $X,Y$ are independent, then we can not get more information about $X$ even if we are given $Y$. Which means

\[H(X)-H(X|Y)=0.\]

If we are given $Y=y$, then the entropy of $X$ will be $H(X|Y=y)$.

\[H(X|Y=y)=-\sum_i \Pr[x_i|Y=y]\log(\Pr[x_i|Y=y]).\]

The formal definition of condition entropy is

\[H(X|Y)=\sum_y \Pr[Y=y]\cdot H(X|Y=y)=-\sum_{x,y}\Pr[X=x,Y=y]\log(\Pr[X=x|Y=y]).\]

Also we have

\begin{align}
	&-\sum_{x,y}\Pr[X=x,Y=y]\log(\Pr[X=x|Y=y]) \notag \\
	=&-\sum_{x,y}\Pr[X=x,Y=y]\left(\log(\Pr[X=x,Y=y])-\log(\Pr[Y=y])\right) \notag \\
	=&-\sum_{x,y}\Pr[X=x,Y=y]\log(\Pr[X=x,Y=y])+ \sum_{x,y}\Pr[X=x,Y=y]\log(\Pr[Y=y]) \notag \\
	=& -\sum_{x,y}\Pr[X=x,Y=y]\log(\Pr[X=x,Y=y])+ \sum_{y}\Pr[Y=y]\log(\Pr[Y=y]) \notag \\
	=&H(X,Y)-H(Y) \notag
\end{align}

The mutual entropy is defined as 

\[I(X,Y)=H(X)-H(X|Y)=H(X)+H(Y)-H(X,Y).\]

\section{Holevo's Theorem} 

A qubit has entropy at most 1. This means you can not communicate more than 1 bit information with 1 qubit. 

If Alice want to transfer with 2 qubit, she encode such that 

\begin{center}
 	\begin{tabular}{|c|c|c|c|}
 		\cline{1-4}
 		00 & 01 & 10 & 11  \\ \cline{1-4}
 		$\vert 0 \rangle$ & $\vert 1 \rangle$ & $\vert + \rangle$ &$\vert - \rangle$  \\\cline{1-4}
 	\end{tabular}
 \end{center}

 Bob can not distinguish non-othorganal. So such transfermation cannot success.  

 But if Alice and Bob obtain a pair of entangle qubits $\vert \psi \rangle_{AB} = \vert 00 \rangle + \vert 11 \rangle.$ Alice can encode the 2 bits by appling measurement $I,\sigma_x,\sigma_z,\sigma_x \sigma_z$ and transfer this qubit to Bob. When bob get qubits, he can perform opreation C-NOT, and measure the first qubit in $\sigma_x$ basis and the second qubit in $\sigma_z$ basis. 

\section{A paradox}
 For entangle qubit, we may have \[H(X,Y)=0,H(X)=1,H(Y)=1.\]

 In this way \[H(X|Y)=H(X,Y)-H(Y)=-1.\]
\end{spacing}

\end{CJK}

\end{document}

